% Options for packages loaded elsewhere
\PassOptionsToPackage{unicode}{hyperref}
\PassOptionsToPackage{hyphens}{url}
%
\documentclass[
]{article}
\usepackage{lmodern}
\usepackage{amssymb,amsmath}
\usepackage{ifxetex,ifluatex}
\ifnum 0\ifxetex 1\fi\ifluatex 1\fi=0 % if pdftex
  \usepackage[T1]{fontenc}
  \usepackage[utf8]{inputenc}
  \usepackage{textcomp} % provide euro and other symbols
\else % if luatex or xetex
  \usepackage{unicode-math}
  \defaultfontfeatures{Scale=MatchLowercase}
  \defaultfontfeatures[\rmfamily]{Ligatures=TeX,Scale=1}
\fi
% Use upquote if available, for straight quotes in verbatim environments
\IfFileExists{upquote.sty}{\usepackage{upquote}}{}
\IfFileExists{microtype.sty}{% use microtype if available
  \usepackage[]{microtype}
  \UseMicrotypeSet[protrusion]{basicmath} % disable protrusion for tt fonts
}{}
\makeatletter
\@ifundefined{KOMAClassName}{% if non-KOMA class
  \IfFileExists{parskip.sty}{%
    \usepackage{parskip}
  }{% else
    \setlength{\parindent}{0pt}
    \setlength{\parskip}{6pt plus 2pt minus 1pt}}
}{% if KOMA class
  \KOMAoptions{parskip=half}}
\makeatother
\usepackage{xcolor}
\IfFileExists{xurl.sty}{\usepackage{xurl}}{} % add URL line breaks if available
\IfFileExists{bookmark.sty}{\usepackage{bookmark}}{\usepackage{hyperref}}
\hypersetup{
  pdftitle={Practica 02},
  pdfauthor={Paula Rozo Coy - Pablo Aguilera Capitan},
  hidelinks,
  pdfcreator={LaTeX via pandoc}}
\urlstyle{same} % disable monospaced font for URLs
\usepackage[margin=1in]{geometry}
\usepackage{graphicx,grffile}
\makeatletter
\def\maxwidth{\ifdim\Gin@nat@width>\linewidth\linewidth\else\Gin@nat@width\fi}
\def\maxheight{\ifdim\Gin@nat@height>\textheight\textheight\else\Gin@nat@height\fi}
\makeatother
% Scale images if necessary, so that they will not overflow the page
% margins by default, and it is still possible to overwrite the defaults
% using explicit options in \includegraphics[width, height, ...]{}
\setkeys{Gin}{width=\maxwidth,height=\maxheight,keepaspectratio}
% Set default figure placement to htbp
\makeatletter
\def\fps@figure{htbp}
\makeatother
\setlength{\emergencystretch}{3em} % prevent overfull lines
\providecommand{\tightlist}{%
  \setlength{\itemsep}{0pt}\setlength{\parskip}{0pt}}
\setcounter{secnumdepth}{-\maxdimen} % remove section numbering

\title{Practica 02}
\author{Paula Rozo Coy - Pablo Aguilera Capitan}
\date{29/2/2020}

\begin{document}
\maketitle

\hypertarget{practica-2---github-y-ciencia-reproducible}{%
\section{Practica 2 - GitHub y Ciencia
Reproducible}\label{practica-2---github-y-ciencia-reproducible}}

\hypertarget{objetivo-general}{%
\section{OBJETIVO GENERAL}\label{objetivo-general}}

Evidenciar los conocimientos adquiridos relacionados principalmente con
el funcionamiento de la libreria R Markdown.

\hypertarget{objetivos-especificos}{%
\section{OBJETIVOS ESPECIFICOS}\label{objetivos-especificos}}

\begin{itemize}
\item
  Reconocer la importancia de la reproducibilidad de la ciencia y los
  elementos del flujo de trabajo reproducible a traves de la comprension
  del articulo \textbf{Ciencia reproducible: qué, por qué, cómo}.
\item
  Hacer uso del repositorio GitHub y entender su importancia
  principalmente en el control de versiones y trabajo integrado por
  medio del articulo \textbf{¿Porqué usar GitHub' Diez pasos para
  disfrutar de GitHub y no morir en el intento}.
\item
  Desarrollar ligereza en el uso de los diferentes argumentos de R
  Markdown para la configuracion de lo que se desea mostrar tanto de
  texto como de codigo en el documento de salida final.
\end{itemize}

\hypertarget{desarrollo-de-la-practica}{%
\section{DESARROLLO DE LA PRACTICA}\label{desarrollo-de-la-practica}}

Para el desarrollo de la presente practica se utiliza el lenguaje de
programación R Core Team (2019), y el documento se genera haciendo uso
de la librería R Markdown. Cabe resaltar que tanto la portada, el
encabezado y el pie de pagina se ajustan en el documento final de word
previo a exportarlo como PDF.

En primer lugar se crea un documento en word nombrado
\textbf{docx\_template\_01} en el que se configura todo el estilo del
documento de salida, adicional a esto se incluyen las referencias de los
dos articulos propuestos por la guia en el archivo de texto
\textbf{refs.bib}.

+\textbf{\emph{Articulo Ciencia reproducible: qué, por qué, cómo. }}

En este apartado se dara respuesta a las preguntas planetadas en la guia
en relación al articulo en mención.

+¿ Que es la ciencia reproducible?

La ciencia reproducible hace referencia a la posibilidad de replicar el
mismo estudio con nuevos datos, es decir, haciendo uso de los mismos
modelos estadisticos, mismos calculos, y obteniendo en teoria la misma
cantidad de resultados representados en el mismo numero de tablas y
figuras. Sin embargo, hoy en dia la mayoria de los articulos y
documentos cientificos no son reproducibles dado que trazar de una
manera clara el proceso de obtencion de los resultados resulta bastante
complicado si no se cuenta con con un codigo - \textbf{Scripts} - que
permita la reproducibilidad de los mismos.

+¿En que casos se consigue 100\% de la reprodicibilidad?

Aunque dificilmente se pueda conseguir 100\% de la roproducibilidad de
un estudio, el texto menciona una serie de elementos que conforma la
transicion hacia la reproducibilidad, la utilizacion inicial de un
codigo (``scripts'') para el análisis de los datos, repositorios de
datos en nube para almacenar, versionar y compratir los datos, la
estructuración de un documento dinamico haciendo uso de herramientas
como Rmarkdown o Ipython, y finalmenye la incorporacion a un repositorio
para conservar un resgistro del desarrollo del proyecto y facilitar la
participacion, en este prden respectivamente, puede llegar a ser el
flujo dinamico antes mencionado para lograr una plena reproducibilidad
de la ciencia. Lo mas complejo en todos estos casos es contar con el
manejo de estas herramientas en los diferentes campos de la ciencia.

+¿Cuales de los beneficios que se sugieren son considerados mas
importantes para nosotros como equipo de trabajo?

Creemos que los beneficios enumerados por los autores del articulo estan
todos interrelacionados de algun modo, sin embargo, hay dos de estos que
consideramos resaltan y denotan la importania de realizar ciencia
reproducible, el primero es que \textbf{``La utilización de código
permite la automatización: ejecución de tareas repetitivas sin
esfuerzo''}, este sin duda permite reproducir los mismos calculos y
analisis no solo reduciendo el esfuerzo si no asegurando que los cambios
que se dan en los resultados por el cambio de metodologias utilizadas
incluso con los mismos datos se reduzca completamente, adicional a esto
es importante mencionar que la automatizacion de la lectura de datos
permite analizarlos incluso cuando no se tiene una comprension del 100\%
de la metodologia utilizada, lo que hace a la ciencia mas accesible y al
alcance de todas las areas de estudio. En segundo lugar se considra
dentro de estos beneficios la \textbf{``Reducción drástica del riesgo de
errores''}, en primer lugar porque auqnue no se garaniza que los
resultados obtenidos una vez se replica el estudio sean correctos si
evita que se sumen mas errores a los ya existentes, es decir indiferente
al numero de veces que se replique el estudio no se existen errores
acumulados por otro lado como lo menciona el texto se abren nuevas
lineas de colaboración y permite que los erreres existentes tengan la
oportunidad de ser corregidos mas rapidamente por alguien que sea aaun
mas experto en el tema en cuestión.

+¿Porque no hace todo el mundo ciencia reproducible?

Como bien lo menciona (CITA DE ARTICULO), desarrollar estudios
plenamente reproducibles implica un esfuerzo incial adicional de
aprendizaje de diferentes areas de estudio, como lo son el manejo de
lenguajes de programación, de bases de datos, de repositorios entre
otras cosas, y este esfuerzo implica dedicar un tiempo considerable a
aprender diciplinas que a la largar pueden no traer un beneficio
economico para quien las aprende y muy por el contrario si requieren de
una inversión; aunque generar ciencia reproducible lo vale, tambien es
importante considerar que no todo el mundo tiene la misma facilidad de
acceso a la herramientas necesarias para llevar a cabo este propósito,
de ahi que otra parte fundamental para que la ciencia sea reproducible,
sea no solo el conocimiento sino que estas herramientas sean de libre
acceso.

+\textbf{\emph{Articulo ¿ Por qué usar GitHub? Diez pasos para disfrutar
de GitHUb y no morir en el intento. }}

En este apartado se tratara de comprobar que quedaron claros los
conceptos mas importantes asociados al respositorio GitHub y como
evidencia del uso y la correcta implementacion del mismo, el script de
este documento mas los documentos anexos necesarios para su correcta
ejecucion puede descargarse del siguiente link
\textbf{\url{https://github.com/PaulaRozo-C/Programacion_avanzada/tree/master/Practica\%2002}}.

Por otro lado comentar a importancia de ciertos conceptos con respeto a
GitHub:

+Repositorio:

+Rama:

+Pull and push:

+Fork:

+\textbf{\emph{Uso de argumentos para la configuración de texto y de
código en R Markdown}}

+Peguen las palabas ``Hola'' y ``mundo'' en una línea nueva:

+Peguen las palabas ``Hola'' y ``mundo'' en la misma línea de código:

+Peguen las palabas ``Hola'' y ``mundo'' en una línea nueva de código y
que no muestren el resultado de R:

+Peguen las palabas ``Hola'' y ``mundo'' en una línea nueva de código y
que no muestren el resultado de R y que no se muestre en el documento de
Word:

\end{document}
